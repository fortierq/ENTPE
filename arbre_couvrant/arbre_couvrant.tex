\documentclass{beamer}
\mode<presentation> { \usetheme{Madrid} }
\setbeamersize{text margin left=18pt,text margin right=18pt}
\setbeamertemplate{footline}{}
\beamertemplatenavigationsymbolsempty

\usepackage{verbatim}
\usepackage{subcaption}

\usepackage{tikz}
\usetikzlibrary{arrows,shapes}
\tikzstyle{vertex}=[circle,draw,minimum size=20pt,inner sep=0pt]
\tikzstyle{selected vertex} = [vertex, fill=red!24]
\tikzstyle{edge} = [draw,thick,-]
\tikzstyle{weight} = [font=\small]
\tikzstyle{selected edge} = [draw,line width=3pt,-,red!100]
\usetikzlibrary{graphs, graphdrawing}
\usegdlibrary{force, layered, circular}
\captionsetup[subfloat]{labelformat=empty}

\pagenumbering{gobble}

\begin{document}

\title{Arbre couvrant de poids minimum}

\begin{frame}
    \titlepage
\end{frame}

% \begin{frame}[fragile]
%     \frametitle{Connexité}
%     Un graphe non orienté est \textbf{connexe} s'il possède un chemin de n'importe quel sommet à n'importe quel autre.
%     \vspace{\baselineskip}

%     \begin{figure}
%         \subfloat[Graphe non connexe]
%         {
%             \tikz \graph [spring layout, nodes={circle, draw}] {
%                 1 -- 3 -- 4 -- 1;
%                 0 -- 5;
%             };
%         }
%         \hspace{1.5cm}
%         \subfloat[Graphe connexe]
%         {
%             \tikz \graph [spring electrical layout, nodes={circle, draw}] {
%                 1 -- 3 -- 4 -- 1 -- 5;
%                 0 -- 5 -- 4 -- 0;
%             };
%         }
%     \end{figure}
% \end{frame}

% \begin{frame}[fragile]
%     \frametitle{Graphe acyclique}
%     Un graphe est \textbf{acyclique} (ou: sans cycle) s'il ne contient pas de cycle.
%     \vspace{\baselineskip}

%     \begin{figure}
%         \subfloat[Graphe contenant un cycle]
%         {
%             \tikz \graph [simple necklace layout,node distance=1.5cm, nodes={circle, draw}] {
%                 2 -- 0 -- 3 -- 2;
%                 0 -- 1;
%                 4;
%             };
%         }
%         \hspace{1.5cm}
%         \subfloat[Graphe acyclique]
%         {
%             \tikz \graph [simple necklace layout,node distance=1.5cm, nodes={circle, draw}] {
%                 2 -- 0 -- 3;
%                 0 -- 1;
%                 4 -- 3;
%             };
%         }
%     \end{figure}
% \end{frame}

% \begin{frame}[fragile]
%     \frametitle{Arbre}
%     \begin{exampleblock}{Définition}
%         Un graphe est un \textbf{arbre} s'il est \textbf{connexe} et \textbf{sans cycle}
%     \end{exampleblock}
%     \pause
%     \vspace{\baselineskip}

%     \begin{alertblock}{Définition}
%         Les graphes ci-dessous sont-ils des arbres?
%     \end{alertblock}

%     \begin{figure}
%         \subfloat[]{
%             \tikz \graph [spring electrical layout,node distance=1cm, nodes={circle, draw}] {
%                 2 -- 0 -- 3 -- 2;
%                 0 -- 1 -- 5;
%                 4 -- 1 -- 3;
%             };
%         }
%         \hspace{1.5cm}
%         \subfloat[]{
%             \tikz \graph [simple necklace layout,node distance=2cm, nodes={circle, draw}] {
%                 2 -- 0 -- 3;
%                 0 -- 1;
%                 4 -- 3;
%             };
%         }
%     \end{figure}
% \end{frame}


% \begin{frame}
%     \frametitle{Arbre couvrant de poids minimal}
%     \begin{exampleblock}{Arbre couvrant}
%         Soit $G$ un graphe pondéré (chaque arête possède un poids).\\
%         Un arbre couvrant de $G$ est un ensemble d'arêtes de $G$ qui forme un arbre et qui contient tous les sommets.
%         Son poids est la somme des poids des arêtes de l'arbre.
%     \end{exampleblock}
%     \pause
%     \vspace{\baselineskip}

%     \begin{exampleblock}{Arbre couvrant de poids minimal}
%         Un arbre couvrant dont le poids est le plus petit possible est appelé un \textbf{arbre couvrant de poids minimal}.
%     \end{exampleblock}
% \end{frame}

% \begin{frame}
%     \frametitle{Arbre couvrant de poids minimal : exemple}
%     \begin{figure}
%         \begin{tikzpicture}[scale=1.8, auto,swap]
%             \foreach \pos/\name in {{(0,2)/a}, {(2,1)/b}, {(4,1)/c},
%                     {(0,0)/d}, {(3,0)/e}, {(2,-1)/f}, {(4,-1)/g}}
%             \node[vertex] (\name) at \pos {$\name$};

%             \foreach \source/ \dest /\weight in {b/a/7, c/b/8,d/a/5,d/b/9,
%                     e/b/7, e/c/5,e/d/15,
%                     f/d/6,f/e/8,
%                     g/e/9,g/f/11}
%             \path[edge] (\source) -- node[weight] {$\weight$} (\dest);
%         \end{tikzpicture}
%     \end{figure}
% \end{frame}

% \begin{frame}
%     \frametitle{Arbre couvrant de poids minimal : exemple}
%     \begin{figure}
%         \begin{tikzpicture}[scale=1.8, auto,swap]
%             \foreach \pos/\name in {{(0,2)/a}, {(2,1)/b}, {(4,1)/c},
%                     {(0,0)/d}, {(3,0)/e}, {(2,-1)/f}, {(4,-1)/g}}
%             \node[vertex] (\name) at \pos {$\name$};

%             \foreach \source/ \dest /\weight in {b/a/7, c/b/8,d/a/5,d/b/9,
%                     e/b/7, e/c/5,e/d/15,
%                     f/d/6,f/e/8,
%                     g/e/9,g/f/11}
%             \path[edge] (\source) -- node[weight] {$\weight$} (\dest);
%             \foreach \source / \dest in {d/a,d/f,a/b,b/e,e/c,e/g}
%             \path[selected edge] (\source) -- (\dest);
%         \end{tikzpicture}
%     \end{figure}
% \end{frame}

% \begin{frame}
%     \frametitle{Arbre couvrant de poids minimal : algorithmes}
%     Il existe deux algorithmes très connus pour trouver un arbre couvrant de poids minimal :
%     \begin{itemize}
%         \item \textbf{Kruskal} : algorithme glouton utilisant un tri des arêtes
%         \item \textbf{Prim} : algorithme construisant l'arbre de proche en proche, similaire à Dijkstra
%     \end{itemize}
% \end{frame}

\begin{frame}
    \frametitle{Algorithme de Kruskal}
    Trier les arêtes par poids croissant.\\
    Commencer avec un arbre T vide (aucune arête).
    \vspace{2\baselineskip}

    Pour chaque arête $a$ par poids croissant:\\
    \hspace{.5cm} Si l'ajout de $a$ ne créé pas de cycle dans T:\\
    \hspace{1cm} Ajouter $a$ à T
\end{frame}

\begin{frame}
    \frametitle{Algorithme de Kruskal}
    \begin{figure}
        \begin{tikzpicture}[scale=1.8, auto,swap]
            \foreach \pos/\name in {{(0,2)/a}, {(2,1)/b}, {(4,1)/c},
                    {(0,0)/d}, {(3,0)/e}, {(2,-1)/f}, {(4,-1)/g}}
            \node[vertex] (\name) at \pos {$\name$};

            \foreach \source/ \dest /\weight in
                {b/a/7, c/b/8, d/a/5, d/b/9, e/b/7, e/c/5,e/d/15, f/d/6,f/e/8, g/e/9,g/f/11}
            \path[edge] (\source) -- node[weight] {$\weight$} (\dest);
            \pause
            \foreach \source / \dest in {d/a,e/c,d/f,b/e,b/a,e/g}
            \path<+->[selected edge] (\source) -- (\dest);
        \end{tikzpicture}
    \end{figure}
\end{frame}

% \begin{frame}
%     \frametitle{Algorithme de Prim}
%     \begin{figure}
%         \begin{tikzpicture}[scale=1.8, auto,swap]
%             \foreach \pos/\name in {{(0,2)/a}, {(2,1)/b}, {(4,1)/c},
%                     {(0,0)/d}, {(3,0)/e}, {(2,-1)/f}, {(4,-1)/g}}
%             \node[vertex] (\name) at \pos {$\name$};

%             \foreach \source/ \dest /\weight in 
%             {b/a/7, c/b/8, d/a/5, d/b/9, e/b/7, e/c/5,e/d/15, f/d/6,f/e/8, g/e/9,g/f/11}
%             \path[edge] (\source) -- node[weight] {$\weight$} (\dest);
%             \pause
%             \foreach \vertex / \fr in {d/1,a/2,f/3,b/4,e/5,c/6,g/7}
%             \path<\fr-> node[selected vertex] at (\vertex) {$\vertex$};
%             \foreach \source / \dest in {d/a,d/f,a/b,b/e,e/c,e/g}
%             \path<+->[selected edge] (\source) -- (\dest);
%         \end{tikzpicture}
%     \end{figure}
% \end{frame}

\end{document}
