\documentclass[draft]{beamer}
\mode<presentation> { \usetheme{Madrid} }
\setbeamersize{text margin left=18pt,text margin right=18pt}
\setbeamertemplate{footline}{}
\beamertemplatenavigationsymbolsempty

\usepackage{hyperref}
\usepackage{verbatim}
\usepackage{subcaption}

\usepackage{tikz}
\usetikzlibrary{arrows,shapes}
\tikzstyle{vertex}=[circle,draw,minimum size=20pt,inner sep=0pt]
\tikzstyle{edge} = [draw,thick,->]
\usetikzlibrary{graphs, graphdrawing}
\usegdlibrary{force, layered, circular}

\pagenumbering{gobble}

\begin{document}

\title{Ordonnancement}

\begin{frame}
    \titlepage
\end{frame}

\begin{frame}[fragile]
    \frametitle{Problème}
    Considérons un ensemble de tâches devant être réalisées.\\
    Chaque tâche a une durée.\\
    Il y a des dépendances entre les tâches : par exemple, une tâche $C$ peut avoir besoin que $A$ et $B$ soient terminées avant de commencer.
    \vspace{2\baselineskip}
    \pause

    Objectif : minimiser le temps total de réalisation des tâches.
\end{frame}

\begin{frame}
    \frametitle{Programmation dynamique}
    On note $w(t)$ la durée d'une tâche $t$.\\
    Soit $t$ une tâche et $T$ l'ensemble des dépendances de $t$ (c'est-à-dire tâches qui doivent finir avant que $t$ commence).
    On veut calculer $d(t)$, la date minimum de fin de $t$.
    \pause
    \vspace{\baselineskip}

    On peut montrer que :
    $$\boxed{d(t) = w(t) + \max_{t' \in T} d(t')}$$
    \pause
    \vspace{\baselineskip}

    La valeur maximum de $d(t)$, pour une tâche quelconque $t$, donne le temps de réalisation total minimum.\\
    Le chemin correspondant est dit \textbf{critique} (toutes les tâches le long de ce chemin doivent être réalisées dès que possible).
\end{frame}

\begin{frame}[fragile]
    \frametitle{Méthode MPM}
    La méthode MPM consiste à considérer un graphe orienté tel que :
    \begin{enumerate}
        \item Les sommets correspond au début d'une tâche.
        \item Chaque arête est une dépendance, avec un poids égal à la durée de la tâche de départ.
    \end{enumerate}
    \pause
    \vspace{\baselineskip}

    De plus on rajoute deux sommets $s$ et $p$ (pour le début et la fin) :
    \begin{enumerate}
        \item $s$ est relié à chaque tâche sans prédécesseur, avec un poids 0.
        \item Chaque tâche $t$ sans successeur est relié à $p$, avec un poids égal à la durée de $t$.
    \end{enumerate}
\end{frame}

\begin{frame}[fragile]
    \frametitle{Méthode MPM}
    \begin{table}[]
        \begin{tabular}{|c|c|c|}
            \hline
            Tâche & Préd & Durée \\ \hline
            A     &      & 5     \\ \hline
            B     &      & 3     \\ \hline
            C     & A    & 2     \\ \hline
            D     & B, C & 4     \\ \hline
            E     & A, C & 3     \\ \hline
        \end{tabular}
    \end{table}
    \pause

    \begin{figure}
        \begin{tikzpicture}[auto,swap]
            \foreach \pos/\name in {{(-2, 1)/s}, {(0,0)/A}, {(0,2)/B}, {(2,1)/C}, {(4,2)/D}, {(4,0)/E}, {(6,1)/p}}
            \node[vertex] (\name) at \pos {$\name$};

            \foreach \source / \dest /\weight in
                {s/B/0, s/A/0, A/C/5, B/D/3, C/D/2, C/E/2, A/E/5, E/p/4, D/p/3}
            \path[edge] (\source) -- node {$\weight$} (\dest);

            \foreach \pos/\v/\k in {{(0,-.5)/0/1}, {(0,2.5)/0/2}, {(1.4,1)/5/3}, {(4,2.5)/7/4}, {(4,-.5)/7/5}, {(6.7,1)/11/6}}
            \onslide<\k-> {\node[color=red] (\v) at \pos {$\v$};};
        \end{tikzpicture}
    \end{figure}

    \color{red}{Dates au plus tôt pour démarrer chaque tâche}
\end{frame}

\begin{frame}[fragile]
    \frametitle{Date au plus tard}
    On a donc trouvé la \textbf{date au plus tôt} pour commencer chaque tâche (la date la plus tôt pour $t$ étant la durée minimale pour réaliser toutes les tâches).
    \vspace{\baselineskip}

    On peut aussi calculer la \textbf{date au plus tard} $f(t)$ d'une tâche $t$ : la date maximum à laquelle on peut démarrer $t$ sans ralentir la durée totale optimale.
    \pause

    Soit $T$ l'ensemble des successeurs de $t$. On peut montrer que :
    $$\boxed{f(t) = \min_{t' \in T} f(t') - w(t)}$$
    \pause

    Ainsi, on peut trouver les dates au plus tard de chaque tâche en utilisant le graphe obtenu précédemment, à l'envers (en partant de $p$).
\end{frame}

\begin{frame}[fragile]
    \frametitle{Méthode PERT}
    La méthode PERT consiste à considérer un graphe orienté tel que :
    \begin{enumerate}
        \item Les arêtes sont les tâches.
        \item Les sommets correspondent aux débuts/fins de tâches.
    \end{enumerate}
\end{frame}

\begin{frame}
    \frametitle{Méthode PERT : exemple}
    Voir vidéo : https://www.youtube.com/watch?v=xAidvykSNXo
\end{frame}

\end{document}
